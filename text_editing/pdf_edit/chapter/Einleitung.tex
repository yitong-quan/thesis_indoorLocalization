
\lhead[\chaptername~\thechapter]{\rightmark}


\rhead[\leftmark]{}


\lfoot[\thepage]{}


\cfoot{}


\rfoot[]{\thepage}

\chapter{Einleitung}
Die Herzfrequenzmessung gehört seit jeher zur gängigsten Methode, um individuelle Belastungsbereiche eines Sportlers zu untersuchen. Dabei genügt heutzutage bereits eine handelsübliche Sportuhr. Somit kann die Herzfrequenz nicht-invasiv in Form eines tragbaren elektronischen Geräts (,,Wearable'') gemessen werden. Die Herzfrequenz stellt allerdings nur den kardiologischen Zustand dar und reicht deshalb nicht aus, um die vollständige Belastbarkeit eines Sportlers zu ermitteln. Hierfür muss unter anderem die Atemgasanalyse herangezogen werden. Bereits Mitte des 19. Jahrhunderts wurden nicht-invasive Atemmessungen, bei denen keine Geräte in den Körper eindringen, mittels Veränderungen der Körperoberfläche durchgeführt. Während erste Untersuchungen anhand der Silhouette von Probanden angestellt wurden, in denen die Brustkorbausdehnung seitlich betrachtet wurde, kann sie heutzutage durch verschiedene Methoden ermittelt werden. Die meisten Atemfrequenzuntersuchungen erfolgen allerdings stationär oder mithilfe einer Atemmaske, wie beispielsweise in der Ganzkörperplethysmographie oder Spirometrie und sind somit für viele Sportarten ungeeignet. Auch die Untersuchung von Säuglingen oder Tieren mit den genannten Methoden birgt viele Schwierigkeiten\cite{heyde}. In den letzten Jahren wurden einige Systeme entwickelt, die auf Basis von Inertialsensoren die Atemfrequenz bestimmen können. Dabei werden Sensoren rund um den Brustkorb angebracht, um dessen Ausdehnung zu erfassen. Anhand dieser Ansätze könnten in Zukunft Messmethoden entwickelt werden, die als Wearable während des Sports oder während der Schlafphase getragen werden können.

Die hier vorgelegte Abschlussarbeit beschreibt Anforderungen an ein System, das auf Intertialsensoren basiert. Dabei werden zuerst Untersuchungen an Probanden durchgeführt, um auf die bestmögliche Position auf dem Brustkorb für ein solches System schließen zu können. Zudem wird untersucht, wie groß ein Apparat dieser Art höchstens sein darf und welche Art von Bewegungen (Rotation oder Translation) dominieren. Des weiteren werden beide Inertialsensortypen (Beschleunigungs- und Drehratensensoren) gegenübergestellt und charakterisiert, inwiefern sie für eine Anwendung von Nutzen sein können. Abschließend wird ein Algorithmus für die Erfassung der Atemfrequenz im stationären und dynamischen Fall vorgestellt und anhand von Messungen validiert und diskutiert.