\chapter{Summary and Outlook}
\label{5}
\section{Summary of the Work}
\label{5_1}
In this master thesis a novel modular reusable and resealable microfluidic platform was developed and characterized. This microfluidic platform was designed for the future work of research bacteria adhesion features at nanofiltration membrane surface. It was integrated with silicone microchannel and nanofiltration membrane with a working pressure of up to 10bars. The silicone membrane with built-in microchannel was attached to a glass slide to form a leak-free sealing. In order to study the bacteria adhesion to the nanofiltration membrane under certain hydrodynamic conditions, the flow rate and the pressure drop in the microchannel were characterized with respect to the system working pressure and permeate flux of the nanofiltration membrane with dead end was measured.\\

To achieve reproducibility of the microchannel and flexible design of the microchannel geometry, a novel processing technique, laser rapid prototyping, was successfully developed. This is a processing technique to pattern different through-cut microchannels in silicone. Since the silicone used in this thesis work does not absorb the laser beam generated by the laser machine (Nd:YAG laser, DPL Smart Marker II from ACI company) in the lab, the indirect ablation is introduced by placing a sacrificial metal layer under the silicone membrane. Besides, the possibility of patterning microchannel with variable depth is verified. The microchannel geometry is designed in CAD to provide the design flexibility. \\

The housing of the microfluidic platform was designed and fabricated. The housing design was improved to achieve better mechanical stability of the integrated glass slide and better sealing performance. The lid of the housing is made from PMMA in order to provide a good vision of the microchannel inside the housing. The microchannel is sealed against the nanofiltration membrane by the compression from the lid. It is fast and flexible to assemble and disassemble this microfluidic platform and the assembling and disassembling does not affect the sealing performance of the microchannel. Furthermore, a test setup was designed and fabricated to pressurize fluid and pump it to the microchannel.\\

At last, pressure and leakage tests of the microfluidic platform were conducted to test the working stability under high pressure. No pressure leakage and fluid leakage presented with the last housing design up to 10 bars. Then the cross flow rate was measured and presented good linearity with respect to the pumping pressure. Two pressure sensors were integrated before and after the microchannel and the pressure data was recorded by Labview through the NI USB-6008 data acquisition module. Then the pressure drop in the connecting tubes as well as in the microchannel was calculated based on the measured flow rate. The permeate flux of the nanofiltration membrane with dead end was also measured under high pressure.\\

To recapitulate, this work develops a novel laser rapid prototyping technique which provides good reproducibility and design flexibility for the patterning of silicone membrane. A modular reusable and resealable microfluidic platform is also designed and fabricated. Then the microfluidic platform is integrated with silicone microchannel patterned by the laser rapid prototyping and connected to the test setup. The working stability is verified under 10bars. At the end the cross flow rate under certain pressure and the permeate flow rate with dead end are measured and the flow rate is characterized with respect to the pumping pressure as well.

\section{Future Work}
\label{5_2}
First of all, the laser rapid prototyping technique can be developed more, mainly in terms of processing microchannel with variable depth. In case the depth of the microchannel is variable, the shear stress at the nanofiltration membrane surface will change with respect to the position along the microchannel, providing a controllable hydrodynamic parameter for the research of bacteria adhesion. Since the processing of microchannel with variable depth will not harm the glass slide, a new transfer process need also be developed as well since the microchannel gasket still need to be through-cut and transferred.\\

Then, the pressure measurement can be improved by using pressure sensors with better accuracy. Furthermore, the pressure sensors are better to be placed right at the inlet and the outlet of the microchannel in order to avoid taking additional pressure into account, which might bring uncertainty and errors. \\

The microfluidic platform needs to be characterized more in terms of cross flow rate under different pressures and permeation flow rate with cross flow. For the calculation of the pressure drop along the microchannel, the pressure loss due to inlet and outlet orifices and the meander geometry of the microchannel should be taken into account in future tests. More hydrodynamic parameters need to be measured and modeled to guarantee that the hydrodynamic conditions in the microchannel are controllable and predictable. Apart from that, the nanofiltration membrane needs to be characterized with salt rejection ratio by measuring the conductivity of the collected solution both from the permeation flow and the cross flow. \\

Since the microfluidic platform designed in this thesis work is aiming at studying the bacteria adhesion features to nanofiltration membranes, suspension with certain bacteria should be pumped into the microchannel for observation and characterization.



































